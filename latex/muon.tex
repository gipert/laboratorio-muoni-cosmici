\documentclass[10pt, oneside, a4paper]{article}   	% {} --> Class: article, report, book, slides
							% [] --> Options: 
							% 11pt = size of the font
							% a4paper, a5paper, letterpaper (default)		
							% titlepage, notitlepage
							% onecolum, twocolumn
							% twoside, oneside = fronteretro o no
							% landscape = foglio orizzontale
								
\usepackage[binding=5mm]{layaureo}			% Layout più espanso preimpostato (bindings = spazio rilegatura)
%\usepackage[options]{geometry}             		% Layout completamente configurabile
							
\pagestyle{plain}					% plain (default)     :	stampa i numeri di pagine a piè di pagina nel mezzo. 
								%		           
								% headings            :	stampa l'intestazione del capitolo attuale e il numero di pagina 
								%                   		nell'intestazione di ogni pagina, il piè di pagina resta vuoto
								% empty               : stampa nulla
								
%\usepackage{syntonly}					% Per controllare la sintassi senza produrre pagine, risparmio tempo
%\syntaxonly						% QUESTO effettivamente dice al compilatore di non produrre pagine

\usepackage{hyperref}					% Collegamenti ipertuestuali (link nell'indice)
\usepackage{graphicx}					% Grafica generale
\usepackage{subfig}					% Subfigures
\usepackage{xcolor}					% Per i colori

\usepackage[italian]{babel}				% Regole di a capo, aggiunta di \ap e \ped per apici e pedici in testo
\usepackage[utf8]{inputenc}				% Specifica la codifica
\usepackage{lmodern}					% Evita "sgranatura" font
\usepackage[T1]{fontenc}				% Risolve i problemi di sillabazione 

\usepackage{microtype}					% Migliora il riempimento delle righe
\usepackage{indentfirst}				% Mette un rientro anche alla prima riga dopo il titolo di paragrafo
%\usepackage[parfill]{parskip}    			% Fa cominciare i paragrafi con una riga vuota piuttosto che con un'indentazione

\usepackage{booktabs}					% Linee separatrici per le tabelle
\usepackage{caption}					% Per mettere la didascalia ben distanziata sopra la tabella
\captionsetup[table]{position=bottom}			% 

\usepackage{amsmath}					% Questo lo ha messo dede
\usepackage{float}					% Questo pure, per fissare la posizione di immagini e tabelle

\usepackage{tikz}
\usepackage{circuitikz}					% Pacchetti per i disegni dei circuiti

\usepackage{amsmath, amsthm, amssymb, amsfonts}		% Matematica

\usepackage{listings}					% Inserimento di codice
\lstset{language=C++}					% Tutti i codici riconosciuti come c++
\lstset{basicstyle=\small\ttfamily,
keywordstyle=\color{blue}\bfseries,
commentstyle=\color{darkgray},
stringstyle=\color{orange}}				% Colori delle righe di codice


\renewcommand{\epsilon}{\varepsilon}			% Varianti lettere greche usate in Europa
\renewcommand{\theta}{\vartheta}			%
\renewcommand{\rho}{\varrho}				%
\renewcommand{\phi}{\varphi}				%

\title{\large{Relazione di Laboratorio Avanzato}\\\Huge{Decadimento dei muoni \\ cosmici in alluminio}}
\date{\today}
%
%
\begin{document}
\maketitle
\thispagestyle{empty}
%
\vspace{4cm}
\begin{abstract}
	Lo scopo dell'esperienza di laboratorio è di caratterizzare e ottimizzare un apparato sperimentale per la rivelazione dei muoni cosmici, consistente nei suoi componenti fondamentali di scintillatori plastici e fotomoltiplicatori, e di studiare il loro decadimento in alluminio anche tramite lo sviluppo di modelli analitici e computazionali. Vengono infine riportati i valori sperimentali delle costanti di decadimento $\tau^+$ e $\tau^-$ delle rispettive particelle muoniche, nonchè il rapporto numerico tra le due popolazioni $N^+/N^-$.
\end{abstract}
\vspace{5cm}
Mattia Faggin \;\,[1122479] \texttt{mattia.faggin.1@studenti.unipd.it} \\
Davide Piras \;\;\;\:[1114287] \texttt{davide.piras@studenti.unipd.it} \\
Luigi Pertoldi \;\,\,[1128283] \texttt{luigi.pertoldi@studenti.unipd.it}
\cleardoublepage
%
\tableofcontents
\clearpage
\listoffigures
\listoftables
%
\clearpage
\section{Apparato sperimentale}
La parte dell'apparato addetta alla rivelazione dei raggi cosmici (figura \ref{appScheme}) consiste di cinque barre (\emph{slab}) di materiale plastico poste parallelamente al terreno ($182 \text{ cm}\times21\text{ cm}\times2.58$ cm), distanti 8 cm tra loro, e altre quattro perpendicolari al terreno poste ai lati ($250\text{ cm}\times21\text{ cm}\times2.58$ cm) a chiusura dell'apparato. Le \emph{slab} centrali sono accoppiate alle estremità a due fotomoltiplicatori e sono intervallate da spessori di alluminio di 2 cm; quelle laterali fanno invece riferimento a un unico fotomoltiplicatore presente a una sola estremità. Uno strato formato da lastre di piombo per un totale di circa 1000 kg ricopre infine la parte superiore dell'apparato: il loro scopo è minimizzare la presenza di altre particelle meno energetiche oltre ai muoni e di frenare questi ultimi al fine di osservarne il decadimento all'interno dell'apparato.
%
\begin{figure}[h]
	\centering
	\subfloat[Schema frontale.]{
\begin{tikzpicture}
	\newcommand\h{0.1}
	\newcommand\dz{0.2}
	\newcommand\dzA{0.1}
	\newcommand\dzS{\dz+2*\h+\dzA}
	\newcommand\dpmt{0.03}
	
	\node at (-0.3,\dz/2) {4};
	\draw[fill=blue!40!white] (0,0-\dpmt) rectangle (1,\dz+\dpmt);
	\draw[fill=gray!20!white] (1,0) rectangle (2,\dz);
	\draw[fill=gray!40!white] (2,0) rectangle (10,\dz);
	\draw[fill=gray!20!white] (10,0) rectangle (11,\dz);
	\draw[fill=blue!40!white] (11,0-\dpmt) rectangle (12,\dz+\dpmt);
	\node at (12.3,\dz/2) {3};

	\draw[fill=black!80!white] (1.5,\dz+\h) rectangle (10.5,\dz+\h+\dzA);
	
	\node at (-0.3,\dz/2+\dzS) {6};
	\draw[fill=blue!40!white] (0,0+\dzS-\dpmt) rectangle (1,\dz+\dzS+\dpmt);
	\draw[fill=gray!20!white] (1,0+\dzS) rectangle (2,\dz+\dzS);
	\draw[fill=gray!40!white] (2,0+\dzS) rectangle (10,\dz+\dzS);
	\draw[fill=gray!20!white] (10,0+\dzS) rectangle (11,\dz+\dzS);
	\draw[fill=blue!40!white] (11,0+\dzS-\dpmt) rectangle (12,\dz+\dzS+\dpmt);
	\node at (12.3,\dz/2+\dzS) {5};

	\draw[fill=black!80!white] (1.5,\dz+\h+\dzS) rectangle (10.5,\dz+\h+\dzA+\dzS);

	\node at (-0.3,\dz/2+\dzS+\dzS) {8};
	\draw[fill=blue!40!white] (0,0+\dzS+\dzS-\dpmt) rectangle (1,\dz+\dzS+\dzS+\dpmt);
	\draw[fill=gray!20!white] (1,0+\dzS+\dzS) rectangle (2,\dz+\dzS+\dzS);
	\draw[fill=gray!40!white] (2,0+\dzS+\dzS) rectangle (10,\dz+\dzS+\dzS);
	\draw[fill=gray!20!white] (10,0+\dzS+\dzS) rectangle (11,\dz+\dzS+\dzS);
	\draw[fill=blue!40!white] (11,0+\dzS+\dzS-\dpmt) rectangle (12,\dz+\dzS+\dzS+\dpmt);
	\node at (12.3,\dz/2+\dzS+\dzS) {7};

	\draw[fill=black!80!white] (1.5,\dz+\h+\dzS+\dzS) rectangle (10.5,\dz+\h+\dzA+\dzS+\dzS);

	\node at (-0.3,\dz/2+\dzS+\dzS+\dzS) {10};
	\draw[fill=blue!40!white] (0,0+\dzS+\dzS+\dzS-\dpmt) rectangle (1,\dz+\dzS+\dzS+\dzS+\dpmt);
	\draw[fill=gray!20!white] (1,0+\dzS+\dzS+\dzS) rectangle (2,\dz+\dzS+\dzS+\dzS);
	\draw[fill=gray!40!white] (2,0+\dzS+\dzS+\dzS) rectangle (10,\dz+\dzS+\dzS+\dzS);
	\draw[fill=gray!20!white] (10,0+\dzS+\dzS+\dzS) rectangle (11,\dz+\dzS+\dzS+\dzS);
	\draw[fill=blue!40!white] (11,0+\dzS+\dzS+\dzS-\dpmt) rectangle (12,\dz+\dzS+\dzS+\dzS+\dpmt);
	\node at (12.3,\dz/2+\dzS+\dzS+\dzS) {9};

	\draw[fill=black!80!white] (1.5,\dz+\h+\dzS+\dzS+\dzS) rectangle (10.5,\dz+\h+\dzA+\dzS+\dzS+\dzS);

	\node at (-0.3,\dz/2+\dzS+\dzS+\dzS+\dzS) {2};
	\draw[fill=blue!40!white] (0,0+\dzS+\dzS+\dzS+\dzS-\dpmt) rectangle (1,\dz+\dzS+\dzS+\dzS+\dzS+\dpmt);
	\draw[fill=gray!20!white] (1,0+\dzS+\dzS+\dzS+\dzS) rectangle (2,\dz+\dzS+\dzS+\dzS+\dzS);
	\draw[fill=gray!40!white] (2,0+\dzS+\dzS+\dzS+\dzS) rectangle (10,\dz+\dzS+\dzS+\dzS+\dzS);
	\draw[fill=gray!20!white] (10,0+\dzS+\dzS+\dzS+\dzS) rectangle (11,\dz+\dzS+\dzS+\dzS+\dzS);
	\draw[fill=blue!40!white] (11,0+\dzS+\dzS+\dzS+\dzS-\dpmt) rectangle (12,\dz+\dzS+\dzS+\dzS+\dzS+\dpmt);
	\node at (12.3,\dz/2+\dzS+\dzS+\dzS+\dzS) {1};
\end{tikzpicture}}\newline
%
%
\subfloat[Schema laterale.]{
\begin{tikzpicture}
	\newcommand\dz{0.2}
	\newcommand\dl{1.5}
	\newcommand\dD{\dz+0.3}
	\newcommand\dDm{-\dz-0.3}
	\newcommand\dDl{\dz+0.4+\dl}
	
	\draw[fill=gray!40!white] (0,0) rectangle (\dz,\dl);
	\draw[fill=gray!40!white] (0,\dl+0.1) rectangle (\dz, 2*\dl+0.1);
	\draw[fill=blue!40!white] (\dz/2,\dl/2) circle (1.05ex); \node at(\dz/2-0.5,\dl/2) {11};
	\draw[fill=blue!40!white] (\dz/2,\dl/2+\dl+0.1) circle (1.05ex); \node at(\dz/2-0.5,\dl/2+\dl+0.1) {14};

	\draw[fill=gray!40!white] (\dz+0.2,\dl-0.05) rectangle (\dz+0.2+\dl,\dl-0.05+\dz); % slab
	\draw[fill=black!80!white] (\dz+0.2,\dl+\dz+0.1-0.05) rectangle (\dz+0.2+\dl,\dl+\dz+0.1-0.05+\dz/2); %Al
	\draw[fill=blue!40!white] (\dz+0.2+\dl/2,\dl+0.05) circle (1.05ex);

	\draw[fill=gray!40!white] (\dz+0.2,\dl-0.05+\dD) rectangle (\dz+0.2+\dl,\dl-0.05+\dz+\dD);
	\draw[fill=black!80!white] (\dz+0.2,\dl+\dz+0.1-0.05+\dD) rectangle (\dz+0.2+\dl,\dl+\dz+0.1-0.05+\dz/2+\dD);
	\draw[fill=blue!40!white] (\dz+0.2+\dl/2,\dl+0.05+\dD) circle (1.05ex);

	\draw[fill=gray!40!white] (\dz+0.2,\dl-0.05+\dD+\dD) rectangle (\dz+0.2+\dl,\dl-0.05+\dz+\dD+\dD);	
	\draw[fill=black!80!white] (\dz+0.2,\dl+\dz+0.1-0.05+\dDm) rectangle (\dz+0.2+\dl,\dl+\dz+0.1-0.05+\dz/2+\dDm);
	\draw[fill=blue!40!white] (\dz+0.2+\dl/2,\dl+0.05+\dD+\dD) circle (1.05ex); \node at (\dz+0.2+\dl/2,\dl+0.05+\dD+\dD+0.5) {2};

	\draw[fill=gray!40!white] (\dz+0.2,\dl-0.05+\dDm) rectangle (\dz+0.2+\dl,\dl-0.05+\dz+\dDm);
	\draw[fill=black!80!white] (\dz+0.2,\dl+\dz+0.1-0.05+\dDm+\dDm) rectangle (\dz+0.2+\dl,\dl+\dz+0.1-0.05+\dz/2+\dDm+\dDm);
	\draw[fill=blue!40!white] (\dz+0.2+\dl/2,\dl+0.05+\dDm) circle (1.05ex);

	\draw[fill=gray!40!white] (\dz+0.2,\dl-0.05+\dDm+\dDm) rectangle (\dz+0.2+\dl,\dl-0.05+\dz+\dDm+\dDm);
	\draw[fill=blue!40!white] (\dz+0.2+\dl/2,\dl+0.05+\dDm+\dDm) circle (1.05ex); \node at (\dz+0.2+\dl/2,\dl+0.05+\dDm+\dDm-0.5) {4};

	\draw[fill=gray!40!white] (0+\dDl,0) rectangle (\dz+\dDl,\dl);
	\draw[fill=gray!40!white] (0+\dDl,\dl+0.1) rectangle (\dz+\dDl, 2*\dl+0.1);
	\draw[fill=blue!40!white] (\dz/2+\dDl,\dl/2) circle (1.05ex); \node at (\dz/2+\dDl+0.5,\dl/2) {12};
	\draw[fill=blue!40!white] (\dz/2+\dDl,\dl/2+\dl+0.1) circle (1.05ex); \node at (\dz/2+\dDl+0.5,\dl/2+\dl+0.1) {13};
\end{tikzpicture}
%
\hspace{2cm}
%
\begin{tikzpicture}
	\newcommand\dz{0.2}
	\newcommand\dl{1.5}
	\newcommand\dD{\dz+0.3}
	\newcommand\dDm{-\dz-0.3}
	\newcommand\dDl{\dz+0.4+\dl}
	
	\draw[fill=gray!40!white] (0,0) rectangle (\dz,\dl);\node at(\dz/2-0.5,\dl/2) {\textcolor{white}{11}};
	\draw[fill=gray!40!white] (0,\dl+0.1) rectangle (\dz, 2*\dl+0.1);

	\draw[fill=gray!40!white] (\dz+0.2,\dl-0.05) rectangle (\dz+0.2+\dl,\dl-0.05+\dz); % slab
	\draw[fill=black!80!white] (\dz+0.2,\dl+\dz+0.1-0.05) rectangle (\dz+0.2+\dl,\dl+\dz+0.1-0.05+\dz/2); %Al
	\draw[fill=blue!40!white] (\dz+0.2+\dl/2,\dl+0.05) circle (1.05ex); 

	\draw[fill=gray!40!white] (\dz+0.2,\dl-0.05+\dD) rectangle (\dz+0.2+\dl,\dl-0.05+\dz+\dD);
	\draw[fill=black!80!white] (\dz+0.2,\dl+\dz+0.1-0.05+\dD) rectangle (\dz+0.2+\dl,\dl+\dz+0.1-0.05+\dz/2+\dD);
	\draw[fill=blue!40!white] (\dz+0.2+\dl/2,\dl+0.05+\dD) circle (1.05ex);

	\draw[fill=gray!40!white] (\dz+0.2,\dl-0.05+\dD+\dD) rectangle (\dz+0.2+\dl,\dl-0.05+\dz+\dD+\dD);	
	\draw[fill=black!80!white] (\dz+0.2,\dl+\dz+0.1-0.05+\dDm) rectangle (\dz+0.2+\dl,\dl+\dz+0.1-0.05+\dz/2+\dDm);
	\draw[fill=blue!40!white] (\dz+0.2+\dl/2,\dl+0.05+\dD+\dD) circle (1.05ex); \node at (\dz+0.2+\dl/2,\dl+0.05+\dD+\dD+0.5) {1};

	\draw[fill=gray!40!white] (\dz+0.2,\dl-0.05+\dDm) rectangle (\dz+0.2+\dl,\dl-0.05+\dz+\dDm);
	\draw[fill=black!80!white] (\dz+0.2,\dl+\dz+0.1-0.05+\dDm+\dDm) rectangle (\dz+0.2+\dl,\dl+\dz+0.1-0.05+\dz/2+\dDm+\dDm);
	\draw[fill=blue!40!white] (\dz+0.2+\dl/2,\dl+0.05+\dDm) circle (1.05ex);

	\draw[fill=gray!40!white] (\dz+0.2,\dl-0.05+\dDm+\dDm) rectangle (\dz+0.2+\dl,\dl-0.05+\dz+\dDm+\dDm);
	\draw[fill=blue!40!white] (\dz+0.2+\dl/2,\dl+0.05+\dDm+\dDm) circle (1.05ex); \node at (\dz+0.2+\dl/2,\dl+0.05+\dDm+\dDm-0.5) {3};

	\draw[fill=gray!40!white] (0+\dDl,0) rectangle (\dz+\dDl,\dl);\node at (\dz/2+\dDl+0.5,\dl/2) {\textcolor{white}{11}};
	\draw[fill=gray!40!white] (0+\dDl,\dl+0.1) rectangle (\dz+\dDl, 2*\dl+0.1);
\end{tikzpicture}}
	
	\caption{Schema dell'apparato di rivelazione. In nero sono rappresentate le barre di alluminio, i numeri etichettano i vari PMT.}
	\label{appScheme}
\end{figure}
%
L'alimentazione è gestita da una serie di partitori di tensione reostatici che permettono di gestire singolarmente il voltaggio dei 14 fotomoltiplicatori. Essi fanno riferimento ad un unico generatore (impostato ad un valore di circa 2100 V, corrispondente alla massima tensione disponibile per ogni fototubo), acceso sempre qualche ora prima di qualsiasi presa dati, in modo tale da poter lavorare nelle condizioni di maggiore stabilità.

I segnali provenienti dal PMT, a polarità negativa, di circa 250 mV di ampiezza e 40 ns di larghezza, vengono formati da dei discriminatori, per poter essere poi utilizzati negli altri moduli.
%
%
\subsection{Prove di luce - buio}
Il primo controllo eseguito è stato quello relativo alla sensibilità degli scintillatori rispetto alla radiazione ambientale: per ogni fotomotiplicatore si sono raccolti dei conteggi in presenza di luce o in assenza di essa (coprendo l'apparato con teli assorbenti, oscurando le finestre presenti nella stanza e spegnendo le luci) per verificare l'approssimativa indipendenza dell'apparato dalle due situazioni. Eventuali infiltrazioni esterne nel rivestimento assorbente delle \emph{slab} sono state riparate con dello scotch isolante nero. 
%
\begin{figure}[h]
	\centering
	\includegraphics[scale=0.6]{img/luce_buio_100s.pdf}
	\caption{Situazione alla prima presa dati: è evidente la presenza di qualche difetto  nell'involucro della \emph{slab} a ridosso del fotomoltiplicatore nr.~12 che permette l'ingresso della radiazione ambientale.}
\end{figure}
%
\subsection{Studio dell'efficienza dei fotomoltiplicatori}
Lo scopo di questa fase consiste nella determinazione dell'efficienza dei 14 fotomoltiplicatori presenti nell'apparato sperimentale. Il comportamento di un generico fotomoltiplicatore, infatti, dipende dalla tensione a cui lo stesso è alimentato: al crescere della tensione, l'efficienza di rivelazione aumenta sino a saturare ad un valore massimo, raggiungendo dunque il regime di \textit{plateau} rispetto ad un contatore di riferimento.
Una volta individuato in quale intervallo di tensioni ciascun fototubo satura, il \emph{set-up} sperimentale ottimale prevede che ciascuno di essi venga alimentato al valore minimo possibile della tensione. 
La condizione di \textit{plateau} è importante per assicurare la massima efficienza di rivelazione possibile; una volta raggiunta, un aumento della tensione di alimentazione coincide solamente con un incremento di rumore.  

L'efficienza di rivelazione corrisponde alla quantità
\[\varepsilon = \frac{T\wedge PMT}{T}\;,\]
dove $T$ corrisponde al numero di eventi rivelati dal cosiddetto \textit{telescopio}, o \textit{trigger}, mentre $T\wedge PMT$ corrisponde alla quantità di eventi rivelati dal fototubo in esame in coincidenza con il telescopio. Il \emph{trigger} prevede l'utilizzo di due \emph{slab} esterne \linebreak ($G$ = grande, $P$ = piccola) poste in coincidenza di volta in volta con alcuni fototubi dell'apparato, differenti da quello momentaneamente esaminato. 

La disposizione del telescopio influenza notevolmente il numero di eventi $T$ tramite i quali si calcola l'efficienza dei diversi fotomoltiplicatori. Ai fini dell'esperimento, l'unica efficienza interessante è relativa alla rivelazione dei raggi cosmici tramite i consueti processi di eccitazione degli atomi del materiale plastico delle \emph{slab}. 
Tuttavia, questa procedura è stata disturbata dalla presenza delle guide di luce di raccordo tra un capo della \emph{slab} e il relativo fototubo: 
avendo a disposizione solo 2 \emph{slab} per formare il telescopio, nel conteggio di trigger entrano, per questioni geometriche, anche quei muoni che attraversano una guida di luce e non le sole \emph{slab}.

In virtù di queste considerazioni, i tracciatori esterni $G$ e $P$ sono stati posizionati sempre parallelamente alle \emph{slab} relative ai PMT esaminati\footnote{$G$ sopra la prima slab, $P$ appoggiata a terra.}: la rivelazione di eventi estranei  
a quelli di interesse \footnote{cioè non passanti per ambo le \emph{slab} ma attraverso una guida di luce} 
non modifica l'andamento della curva di efficienza, ma solamente i valori assunti dalla stessa, permettendo comunque di individuare il raggiungimento del regime di \emph{plateau}.

In base al PMT studiato, il telescopio corrisponde ad una delle seguenti configurazioni:
%%%%%%%%%%%%%%%%%%
\begin{itemize}
	\item \textbf{configurazione 1}: per lo studio dei PMT numero 1, 2, 5, 6, 7, 8, 9 e 10 i tracciatori $G$ e $P$ sono messi in coincidenza con i PMT numero 3 e 4 (relativi alla \emph{slab} più bassa), cioè \[T = G\wedge P \wedge PMT3 \wedge PMT4 \ ;\]
%
	\item \textbf{configurazione 2}: per lo studio dei PMT numero 3 e 4 i tracciatori $G$ e $P$ sono messi in coincidenza con i PMT numero 1, 2, 5, 6, cioè \[T = G\wedge P \wedge PMT1 \wedge PMT2 \wedge PMT5 \wedge PMT6 \ ; \]
%
	\item \textbf{configurazione 3}: le \emph{slab} laterali sono state momentaneamente tolte dalla loro posizione originale per essere appoggiate, sovrapposte nella stessa direzione e alternate in verso, su sgabelli di eguale altezza. In questo modo, per lo studio dei PMT 11 e 12, posizionati nello stesso verso, i tracciatori $G$, appoggiato sopra la prima delle quattro slab, e $P$, appoggiato a terra, sono stati messi in coincidenza con i PMT numero 13 e 14, anch'essi orientati nello stesso verso, opposto ai precedenti. In questo caso: \[T = G\wedge P \wedge PMT13 \wedge PMT14 \ ;\]
%
	\item \textbf{configurazione 4}: questa configurazione coincide con la numero 3, a patto di invertire i ruoli dei PMT 11 e 12 con quelli dei PMT 13 e 14. 
\end{itemize}
%%%%%%%%%%%%%%%%%%%
In appendice sono riportati i grafici relativi allo studio di tutti i 14 fotomoltiplicatori, sia per quanto riguarda la stima delle efficienze da vicino e da lontano, sia per quanto riguarda i conteggi da vicino.

L'efficienza di rivelazione è definita come segue:
$$\epsilon=\frac{T\wedge PMT}{T}$$
Al numero di eventi in coincidenza $T\wedge PMT$ è assegnata un'incertezza seguente la distribuzione binomiale, di cui si conosce la formula della varianza.  
Si ha allora che 
\[\sigma_{T\wedge PMT} = \sqrt{T \cdot \varepsilon \cdot  (1-\varepsilon)}\]
dove si è assunta come probabilità dell'evento favorevole il valore sperimentale dell'efficienza stessa $\varepsilon$.
L'errore sull'efficienza è dunque:
$$\sigma_\epsilon=\sqrt{ \left(\frac{\partial\epsilon}{\partial (T\wedge PMT)}\sigma_{T\wedge PMT}\right)^2 + \left(\frac{\partial \epsilon}{\partial T}\sigma_T\right)^2 }=\sqrt{\frac{\epsilon}{T}}$$
dove $\sigma_T = \sqrt{T}$.

%
\subsection{Misure di differenze temporali}
L'acquisizione della curva di decadimento dei muoni necessita di una procedura per la misura di intervalli temporali tramite un modulo Time to Amplitude Converter (TAC) e un modulo Analog to Digital Converter (ADC) per la conversione digitale. Le proprietà desiderate della catena elettronica devono essere opportunamente verificate prima di eseguire una procedura di calibrazione che permetta di associare un canale del multicanale ad un ben preciso valore temporale.

Inizialmente si è verificato il corretto comportamento della catena TAC + ADC rispetto a una distribuzione uniforme di ritardi temporali, ovvero generando due treni di segnali scorrelati da mandare in input alle entrate di START e STOP del TAC; ci si aspetta dalla raccolta di un numero statisticamente rilevante di dati una distribuzione perfettamente piatta. L'apparato dispone di due modelli differenti di TAC, nr.~437 e nr.~467: in figura \ref{rumore1} è possibile osservare i risultati di una prima acquisizione.
%
\begin{figure}[h]
	\centerline{
		\subfloat[Nr.~437]
		{\includegraphics[scale=0.374]{img/bkg_sx.pdf}}
		\subfloat[Nr.~467]
		{\includegraphics[scale=0.374]{img/bkg_dx1.pdf}}}
		\caption{Prima analisi dello spettro di ritardi temporali generati uniformemente; si nota nella figura (b) la presenza di un picco di eventi indesiderato nei pressi del canale 700.}
		\label{rumore1}
\end{figure}
%

Il TAC nr.~437 non sembra presentare problemi, tranne che per il picco all'inizio dello spettro di cui non si è scoperta l'origine, ma che non influisce sulla raccolta dati. Il nr.~467 invece esibisce un picco di eventi inaspettato riconducibile a un problema di cattivo accoppiamento di impedenze di uscita del TAC e di ingresso dell'ADC; dopo aver modificato l'impedenza di uscita del TAC tramite un connettore BNC a ``T'' chiuso da un lato su 50 $\Omega$ lo spettro risulta uniforme (figura \ref{rumore2}).

%
\begin{figure}[]
	\centering
	\includegraphics[scale=0.45]{img/bkg_dx2.pdf}
	\caption{Spettro finale per il TAC nr.~467.}
	\label{rumore2}
\end{figure}
%
Dopo questa prima fase si è verificato che il TAC nr.~467 non erogasse più di 5 V (il massimo valore di tensione convertibile dall'ADC) in corrispondenza di un ritardo di 20 $\mu$s e nello stesso modo si è impostato il fondoscala del nr.~437. Si è proceduto quindi a verificare la linearità della catena elettronica TAC + ADC e ottenere quindi una calibrazione temporale dello spettro registrato dal multicanale: tramite due copie del segnale di un impulsatore opportunamente discriminato fornite in input al TAC, separate da una distanza temporale variabile nel range $[0 \ \mu \mbox{s},20 \ \mu \mbox{s}]$, si sono registrati i relativi valori di tensione erogati e il canale corrispondente (vedi appendice B, figure \ref{time_tens} e \ref{chan_time}).
%
%
%\begin{figure}[h]
%	\centerline{
%		\includegraphics[scale=0.8]{img/time_tens_437.pdf}}
%	\caption{Valori di tensione erogati del TAC nr.~437.}
%	\label{tt437}
%\end{figure}
%
%\begin{figure}[h]
%	\centerline{
%		\includegraphics[scale=0.8]{img/time_tens_467.pdf}}
%	\caption{Valori di tensione erogati del TAC nr.~467.}
%	\label{tt467}
%\end{figure}
%
%\begin{figure}[h]
%	\centerline{
%		\includegraphics[scale=0.8]{img/chan_time_437.pdf}}
%	\caption{Calibrazione canale - tempo per il TAC nr.~437.}
%	\label{ct437}
%\end{figure}
%
%\begin{figure}[h]
%	\centerline{
%		\includegraphics[scale=0.8]{img/chan_time_467.pdf}}
%	\caption{Calibrazione canale - tempo per il TAC nr.~467.}
%	\label{ct467}
%\end{figure}
%
Con i fondoscala impostati come descritto in precedenza, i TAC nr.~437 e nr.~467 erogano rispettivamente per valori del ritardo compresi tra 60 ns e 20 $\mu$s segnali nei range di tensione $[0.211 \ \mbox{V}, 5.070 \ \mbox{V}]$ e $[0.003 \ \mbox{V}, 4.670 \ \mbox{V}]$. Ciò introduce una complicazione nell'utilizzo del TAC nr.~467: infatti  0.003 V sono troppo pochi per essere rivelati dall'ADC (la soglia inferiore di lettura è all'incirca di 30 mV, corrispondente ad un ritardo di ben 650 ns) e quindi c'è una vasta classe di ritardi che non è visibile dalla catena elettronica. Per risolvere il problema è possibile ad esempio aggiungere sistematicamente ai segnali in input un ritardo temporale noto superiore o uguale a 650 ns affinchè l'ADC sia in grado di rivelare qualsiasi valore del ritardo, per quanto piccolo, e poi tenerne conto in fase di calibrazione degli spettri. 

Per i risultati esposti di seguito, è stato utilizzato solamente il TAC nr.~437.
%
%
\section{Operazioni preliminari alla presa dati}
\subsection{Conteggi in coincidenza}
La fase di studio successiva consiste nel raccogliere i conteggi in coincidenza tra varie \emph{slab} e conseguentemente ottimizzare i valori di tensione applicata ai fotomoltiplicatori al fine di ottenere valori uniformi tra loro e in linea con le previsioni teoriche (vedi appendice A). 
Innanzitutto, si sono scelti i seguenti valori di tensione dei PMT: 
\begin{table}[h]
		\centering
	\begin{tabular}{ccc}
		\toprule
		PMT	&	HV [V]	&	Frequenza [Hz]\\	
		\midrule
		1	&	1860	&	109	\\
		2	&	1940	&	91	\\
		3	&	1930	&	100	\\
		4	&	1920	&	121	\\
		5	&	1890	&	121	\\
		6	&	1950	&	105	\\
		7	&	1770	&	123	\\
		8	&	1890	&	173	\\
		9	&	1670	&	128	\\
		10	&	1810	&	101	\\
		11	&	1750	&	124	\\
		12	&	1840	&	82	\\
		13	&	1960	&	86	\\
		14	&	1800	&	197	\\
		\bottomrule
	\end{tabular}
	\caption{Tensione finale applicata ai fotomoltiplicatori, L'incertezza associata a ciascuna frequenza è di 1 Hz. Le tensioni scelte garantiscono la condizione di \emph{plateau} da distante rispetto al fototubo.}
	\label{HV_counts}
\end{table}
Si sono quindi raccolti i dati di varie coincidenze, prima senza e poi sovrapponedo il piombo, per testare la diminuzione di eventi rivelati in presenza delle lastre, e verificare che la geometria dell'apparato fosse consistente con il numero di eventi raccolti. I simboli L e R indicano rispettivamente L $= 14 \wedge 11 $ e R $= 12 \wedge 13$, e insieme alla coppia di PMT 3 e PMT 4 (cioè, S5) formano la cosiddetta \emph{scatola}, utile per i circuiti di START e STOP, analizzati nel paragrafo successivo.
%
\begin{table}[H]
	\centering
	\begin{tabular}{ccc}
		\toprule
		Coincidenze						&	Frequenza senza Pb [Hz] & Frequenza con Pb [Hz]	\\
		\midrule
		S1 (PMT1 $\wedge$ PMT2)					& $60.6 \pm 0.8$	& $52.0 \pm 0.7$ \\
		S2 (PMT9 $\wedge$ PMT10)				& $64.1 \pm 0.8$	& $60.0 \pm 0.8$ \\
		S3 (PMT7 $\wedge$ PMT8)					& $77.2 \pm 0.9$	& $75.1 \pm 0.9$ \\
		S4 (PMT5 $\wedge$ PMT6)					& $63.4 \pm 0.8$	& $61.5 \pm 0.8$ \\
		S5 (PMT3 $\wedge$ PMT4)					& $61.1 \pm 0.8$	& $60.3 \pm 0.8$ \\
		S1 $\wedge$ S2 						& $39.0 \pm 0.6$	& $36.2 \pm 0.6$ \\
		S2 $\wedge$ S3 						& $37.9 \pm 0.6$    	& $36.4 \pm 0.6$ \\
		S3 $\wedge$ S4 						& $37.5 \pm 0.6$	& $36.3 \pm 0.6$ \\
		S4 $\wedge$ S5 						& $36.0 \pm 0.6$	& $35.3 \pm 0.6$ \\
		S1 $\wedge$ S3 						& $29.2 \pm 0.5$	& $26.9 \pm 0.5$ \\
		S1 $\wedge$ S4 						& $22.9 \pm 0.5$	& $21.0 \pm 0.5$ \\
		S1 $\wedge$ S5 						& $17.7 \pm 0.4$	& $16.0 \pm 0.4$ \\
		S1 $\wedge$ S2 $\wedge$ S3				& $29.1 \pm 0.5$	& $27.5 \pm 0.5$ \\
		S2 $\wedge$ S3 $\wedge$ S4				& $28.7 \pm 0.5$	& $27.8 \pm 0.5$ \\
		S3 $\wedge$ S4 $\wedge$ S5				& $28.6 \pm 0.5$	& $27.6 \pm 0.5$ \\
		S1 $\wedge$ S2 $\wedge$ S3 $\wedge$ S4			& $21.8 \pm 0.5$	& $21.8 \pm 0.5$ \\
		S2 $\wedge$ S3 $\wedge$ S4 $\wedge$ S5			& $20.8 \pm 0.5$	& $21.0 \pm 0.5$ \\
		S1 $\wedge$ S2 $\wedge$ S3 $\wedge$ S4 $\wedge$ S5	& $16.7 \pm 0.4$	& $16.7 \pm 0.4$ \\	
		S1 $\wedge$ S2 $\wedge$ L				& $8.9 \pm 0.3$		& $8.8 \pm 0.3$	\\
		S1 $\wedge$ S2 $\wedge$ R				& $8.4 \pm 0.3$		& $7.7 \pm 0.3$	\\
		S3 $\wedge$ S4 $\wedge$ L				& $7.5 \pm 0.3$		& $8.0 \pm 0.3$	\\
		S3 $\wedge$ S4 $\wedge$ R				& $9.2 \pm 0.3$ 	& $9.3 \pm 0.3$	\\
		S1 $\wedge \ (\overline{\mbox{S5}\vee \mbox{L} \vee \mbox{R}})$ (veto)	& $51.3 \pm 0.7$	& $43.2 \pm 0.7$	\\
		\bottomrule
	\end{tabular}
	\caption{Coincidenze. 
	Si noti che nell'ultima misura si è utilizzato il veto per negare il segnale, 
	in quanto esso è risultato più efficiente dell'uscita negata dei discriminatori 
	nell'individuare il segnale voluto: talvolta, infatti, l'uscita negata dei discriminatori non forniva in uscita il segnale desiderato, 
	probabilmente a causa di un malfunzionamento dei moduli o di un problema nella temporizzazione dei segnali. Per ovviare a ciò, con riferimento all'ultima coincidenza,
	si è messo il segnale $S5\vee L\vee R$ (opportunamente temporizzato)
	nell'ingresso di veto del discriminatore del 
	segnale $S1$: qualora il veto fosse stato attivo, l'output 
	sarebbe stato nullo.}
	\label{coincidenze}
\end{table}
%
\subsection{Costruzione del circuito}
A questo punto, dopo tutte le verifiche necessarie, e dopo aver ripetuto le prove di luce-buio\footnote{Si è notato un eccesso di segnale proveniente da una zona della \emph{slab} superiore, subito risolto con dello scotch isolante.}, si sono costruiti i circuiti di START e di STOP, per la rivelazione dei muoni cosmici. Si rende però necessario fare una premessa: potendo utilizzare nella costruzione del circuito segnali negati o di veto, dopo alcune prove si è optato per la seconda possibilità, avendo cura di temporizzare bene le coincidenze, ossia facendo in modo che il segnale di veto rendesse effettivamente cieco l'apparato. In questo passaggio il segnale di veto è stato ritardato e allargato cosicchè contenesse il segnale del PMT in coincidenza.

Per il segnale di START, corrispondente al muone che entra dall'alto e non esce dalla \emph{scatola}, si è partiti considerando varie possibilità; il miglior compromesso, che generasse un segnale sufficientemente pulito ma con una rate di conteggio non troppo bassa, si è rivelato essere il segnale S1 $\wedge$ S2  $\wedge$ $\overline{\mbox{S5}\vee \mbox{L} \vee \mbox{R}}$ , la cui rate in coincidenza è 6.6 Hz.
Il segnale di STOP corrisponde invece all'avvenuto decadimento e rivelazione dell'elettrone emesso, ed è stato scelto essere S3 $\veebar$ S4  $\wedge$ $\overline{\mbox{S1} \vee \mbox{S5}\vee \mbox{L} \vee \mbox{R}}$, con rate 38.1 Hz (schemi in figura \ref{scemi}).

\begin{figure}[H]
	\centering
	 \subfloat[START.]
 {\begin{circuitikz}
      \draw 
      (0.5,10.97)  to[short,o-] (1.5,10.97)
      (0.5,11.53)  to[short, o-] (1.5,11.53)
      (2.88,11.25) node[american and port] {}
		   node[left=6pt] {\tiny{AND}}
      (0.5,10.97) node[below] {S2}
      (0.5,11.53) node[above] {S1}
      (3.03,11.25) to [short,-o] (5.3, 11.25)
      (2.5,6.97)  to[short,o-] (3.5,6.97)
      (2.3,7.25)  to[short,o-] (3.93,7.25)
      (2.5,7.53)  to[short, o-] (3.5,7.53)
      (2.5,6.97) node[below] {L}
      (2.5,7.53) node[above] {R}
      (2,7.25) node[] {S5}     
      (4.88,7.25) node[american or port] {}
                  node[left=6pt] {\tiny{OR}}
      (5.03,7.25) to[short] (5.03, 11.25)
      (4.85, 10.60) node[rotate=90] {\scriptsize{VETO}}
      ;
    \end{circuitikz}}
    \hspace{1.5 cm}
\subfloat[STOP.]
{\begin{circuitikz}
      \draw 
      (0.5,10.97)  to[short,o-] (1.5,10.97)
      (0.5,11.53)  to[short, o-] (1.5,11.53)
      (2.88,11.25) node[american or port] {}
		   node[left=6pt] {\tiny{OR}}
      (3.4,9.15) node[american and port] {}
		 node[left=6pt] {\tiny{AND}}
      (1, 8.87) to[short] (1, 10.97)
      (1, 8.87) to[short] (2.03, 8.87)
      (1.3, 9.43) to[short] (1.3, 11.53)
      (1.3, 9.43) to[short] (2.03, 9.43)
      (3.55, 9.15) to[short] (4.2, 9.15)
      (4.2, 9.15) to[short] (4.2,7.73)
      (0.5,10.97) node[below] {S4}
      (0.5,11.53) node[above] {S3}
      (3.03,11.25) to [short,-o] (5.3, 11.25)
      (2.5,6.97)  to[short,o-] (3.5,6.97)
      (2.3,7.15)  to[short,o-] (3.93,7.15)
      (2.3,7.35)  to[short,o-] (3.93,7.35)
      (2.5,7.53)  to[short, o-] (3.5,7.53)
      (2.5,6.97) node[below] {L}
      (2.5,7.53) node[above] {R}
      (1.9,7.07) node[] {S1}     
      (1.9,7.43) node[] {S5}   
      (4.88,7.25) node[american or port] {}
		  node[left=6pt] {\tiny{OR}}
      (5.03,7.25) to[short] (5.03, 11.25)
      (4.85, 10.60) node[rotate=90] {\scriptsize{VETO}}
           ;
    \end{circuitikz}}

	\caption{Schema dei circuiti di START e di STOP. Si noti che nel circuito di STOP si è utilizzato un segnale di XOR, ossia il veto ha una componente che rende cieco l'apparato quando entrambe le \emph{slab} S3 e S4 rivelano un evento; ciò viene utilizzato per cercare di incrementare la probabilità che il segnale di STOP non sia dato da un evento spurio, assumendo che una volta decaduto il muone l'elettrone si fermi alla prima \emph{slab} che incontra.}
	\label{scemi}
\end{figure}

\section{Analisi dati preliminare}
Dopo aver impostato il circuito come illustrato precedentemente, si è proceduto all'acquisizione dei dati relativi al ritardo che intercorre tra il passaggio del muone cosmico attraverso l'apparato di START e l'uscita dell'elettrone di decadimento attraverso lo STOP.

I dati consistono di tre acquisizioni, per un totale di circa 75000 eventi, riguardanti due popolazioni differenti di muoni\footnote{Dati presi da K.A. Olive \emph{et al.} (Particle Data Group), Chin. Phys. C, \textbf{38}, 090001 (2014) and 2015 update.}:
\begin{itemize}
	\item i $\mu^+$, tali che $\mbox{Br}(\mu^+ \to e^+\nu_e\overline{\nu}_\mu)\simeq 100 \%$ e aventi una vita media \linebreak $\tau^+=(2.1969811 \pm 0.0000022)\;\mu\mbox{s}$;
\item i $\mu^-$, i quali però si comportano in maniera differente. Essi, infatti, oltre a decadere nel canale $\mu^-\to e^-\overline{\nu}_e\nu_\mu$, possono sostituirsi ad un elettrone degli atomi del mezzo, formando un atomo muonico in uno stato metastabile che favorisce il processo di cattura $\mu^-+p\to n+\nu_\mu $, con $p$ protone del nucleo del mezzo. Questa implica una diminuzione della vita media per i $\mu^-$ al valore $\tau^-=(0.88 \pm 0.01) \ \mu \mbox{s}$.
\end{itemize}

I vari contributi possono essere convogliati nella seguente formula, che modellizza lo spettro che si ottiene tramite l'ADC:
 \[  \frac{dN}{dt} = A^+ e^{-t/\tau^+} + A^- e^{-t/\tau^-} + \mbox{\emph{bkg\;.}} \]
Queste due popolazioni vanno ovviamente considerate in maniera separata, sapendo che lo spettro raccolto prevede una prima parte relativa ai $\mu^-$ (per $t<4\tau^-$), una seconda parte relativa ai soli $\mu^+$ (per $4\tau^-<t<4\tau^+$) ed infine puro fondo (per $t>4\tau^+$), distinguibili grazie alla calibrazione temporale del sistema TAC+ADC.
L'analisi dati si svolge in base ai seguenti step:
%%%%%%%%%%%%%%%%%%%%%%%%%%%%%%
\begin{enumerate}
	\item si definisce un istogramma con 4096 canali, nel quale inserire tutta la statistica raccolta, assegnando a ciascun bin un contenuto pari alla somma dei contenuti del medesimo canale delle varie acquisizioni (in questo caso 3);
	\item si stima il fondo come la media tra i contenuti dei bin a partire da quello che, in base alla calibrazione, coincide con $t=4\tau^+$, e tale valore si sottrae al contenuto di ciascun bin;
	\item si definisce la funzione $f(t)=Ae^{-t/\tau}$, dove la variabile indipendente $t$ e $\tau$ si riferiscono a grandezze in canali; quindi si esegue un fit dell'istogramma nella regione dei soli $\mu^+$. Questa operazione permette di ricavare sia la vita media $\tau^+$ che l'abbondanza $A^+$ dei $\mu^+$; 
	\item al contenuto di ciascun canale dell'istogramma si sottrae il valore assunto dalla funzione suddetta in corrispondenza del centroide di ciascun canale, in modo tale da ricavare la distribuzione dei soli $\mu^-$;
	\item si effettua il fit dell'istogramma risultante, ricavando i parametri relativi ai $\mu^-$.
\end{enumerate}
%%%%%%%%%%%%%%%%%%%%%%%%%%%%%%
Questo procedimento presenta tuttavia un aspetto critico, da mettere a punto nella seconda parte dell'esperienza. 
Il fit della regione relativa ai $\mu^+$ riguarda canali dell'istogramma con bassa statistica e questo provoca una notevole fluttuazione dei risultati, in particolare al variare del numero di bin. 

Per quanto riguarda il rapporto numerico $N^+/N^-$ tra le due popolazioni, la stima è stata ottenuta tramite la seguente formula: 
\[\frac{A^+}{A^-}= \bigg ( \frac{\tau^+}{\tau^-} \bigg )  ^{-1} \frac{N^+}{N^-} \bigg ( \frac{\Gamma_d}{\Gamma_d+\Gamma_c} \bigg )^{-1}e^{t^*\left(\frac{1}{\tau^+}-\frac{1}{\tau^-}\right)}\]
dove $\Gamma_d$ e $\Gamma_c$ sono rispettivamente le larghezze parziali di decadimento e cattura per entrambe le popolazioni, mentre il termine esponenziale è stato aggiunto per tenere conto del fatto che l'apparato è cieco per un tempo $t^* \sim 150$ ns precedente allo zero.
%
\begin{figure}[H]
\centering
\subfloat[][{Eventi raccolti.}]
{\includegraphics[width=1\textwidth]{img/Spettro_rebin12.pdf}} \\
\subfloat[][{Istogrammi sovrapposti}.]
{\includegraphics[width=1\textwidth]{img/histo_sovrapposti.pdf}} \\
\caption{Istogrammi dei dati raccolti e analizzati secondo quanto spiegato precedentemente, con un coefficiente di rebin 12.}
\label{fig::histo}
\end{figure}
%
In figura \ref{fig::histo} viene riportato il grafico degli eventi raccolti, dove l'istogramma ha un numero di bin ridotto di un fattore 12 rispetto a quello originale, appunto per ridurre le fluttuazioni statistiche in maniera ottimale. I risultati derivanti da questa analisi sono riportati in tabella \ref{results}.
\begin{table}[H]
	\centering
	\begin{tabular}{ccccc}
		\toprule
		& Misura [$\mu$s]	&	Valore atteso [$\mu$s]	 & Compatibilità & Differenza 	\\	
		\midrule
		$\tau^+$	&	$2.20\pm 0.05$ & $2.1969811 \pm 0.0000022$ & 0.16 & 0.0072134 $\mu$s	\\
		$\tau^-$	&	$0.870\pm 0.007$ & $0.88 \pm 0.01$ & 0.80 & 0.01 $\mu$s	\\
		$N^+/N^-$	&	$1.5 \pm 0.1$ & $1.261 \pm 0.009$ & 1.95 & 0.20	\\
		\bottomrule
	\end{tabular}
	\caption{Risultati dei fit (rebin 12).}
	\label{results}
\end{table}
\section{Simulazione MonteCarlo e analisi dati}

Lo scopo di questa fase dell'esperimento consiste nella ricerca e messa a punto di un
metodo sperimentale di analisi degli spettri raccolti. Il lavoro si suddivide in due step:
\begin{enumerate}
 \item sviluppare un codice di simulazione MonteCarlo (MC) in ambiente ROOT nel quale si simulano molti spettri di decadimento dei muoni cosmici in un range utile di circa 3900 canali, in accordo con gli spettri acquisiti. Gli istogrammi simulati vengono quindi utilizzati per testare diversi metodi di fit per l'esrazione dei parametri fisici;
 \item utilizzare il metodo di fit stabilito nello step precedente per estrarre $\tau^+$, $\tau^-$ e $R=N^+/N^-$ dagli spettri raccolti.
\end{enumerate}
 
\subsection{Simulazione del fondo}
La prima versione del codice MC prevede la simulazione di spettri costituiti da solo rumore (che da ora in avanti chameremo anche \textit{baseline}), attraverso la generazione uniforme di numeri casuali (\lstinline{TRandom3::Uniform(double Begin,double End)}). 
I metodi testati per la stima della \textit{baseline} sono i seguenti:
\begin{itemize}
 \item media aritmetica dei contenuti dei bin dell'istogramma \lstinline{baseline};
 \item fit con il metodo del $\chi^2$ utilizzando la funzione \lstinline{pol0} (\lstinline{baseline.Fit("pol0","R")});
 \item metodo della maximum likelihood utilizzando la funzione \lstinline{pol0} (\lstinline{baseline.Fit("pol0","LR")}).
\end{itemize}

\begin{figure}[H]
 \includegraphics[scale=0.75]{img/baseline_sim_500.pdf}
 \caption{Distribuzioni delle stime della \textit{baseline} con i tre metodi differenti, sia per i valori sia per gli errori (500 simulazioni). La \textit{baseline} generata in queste simulazioni ha altezza 10.}
 \label{fig::baseline_500}
\end{figure}
In figura \ref{fig::baseline_500} è riportato un esempio delle distribuzioni dei valori e degli errori derivanti da simulazioni multiple, stimati con i tre metodi. La linea rossa verticale nei grafici in alto indica il valore vero della \textit{baseline}, ossia il valore con cui sono stati simulati gli spettri, quindi il valore che dovrebbe essere restituito dall'analisi. Come si può notare, la distribuzione dei valori estratta con fit \lstinline{pol0} di tipo $\chi^2$ non è centrata attorno al valore atteso. In questo caso, infatti, la baseline è sistematicamente sottostimata, poichè:
\begin{enumerate}
 \item il metodo dei minimi quadrati non funziona correttamente per il fit di un istogramma con bin a bassa statistica, per i quali l'assegnazione di errore di tipo poissoniano (cioè la radice del conteggio) presenta delle criticità, in particolare per contenuti nulli;
 \item il fit di una \lstinline{pol0} corrisponde esattamente ad una media pesata, nella quale i bin con statistica minore hanno un peso maggiore, con conseguenti valori sottostimati della \textit{baseline}
 \footnote{Media pesata ed errore:
 $$\bar{B}=\frac{\sum_{i=1}^N B_i/\sigma_i^2}{\sum_{i=1}^N 1/\sigma_i^2};
 \hspace{1cm} \sigma_{\bar{B}}=\frac{1}{\sum_{i=1}^N1/\sigma_i^2}
 $$}
 .
\end{enumerate}
In virtù di queste considerazioni, il metodo di fit del $\chi^2$ è stato dunque scartato. 
Gli spettri relativi agli errori statistici sulla \textit{baseline} sono utili per controllare che le distribuzioni dei valori non siano troppo disperse. In altre parole, si effettua il confronto tra la $\sigma$ delle gaussiane dei valori e il centroide $\overline{\sigma_{B}}$ delle gaussiane degli errori, controllando che $\sigma\lesssim \overline{\sigma_{B}}$, condizione soddisfatta nelle simulazioni in esame.



 
%Un'ultima nota riguarda un fatto di cui, per il momento, non si è riusciti a dare una spiegazione: si è infatti raccolto un quarto file di dati, che, pur dando valori dei parametri consistenti con quelli presentati in tabella \ref{results}, non risultano invece compatibili con gli altri 3 set di misure. A questo dubbio si potrà dare una risposta solo una volta tornati in laboratorio.
%
%
%
\appendix
%\setcounter{secnumdepth}{0}
\section{Previsioni teoriche}
Nei paragrafi successivi sono riportati due diversi approcci, uno analitico e uno computazionale, per tentare di stimare la percentuale di eventi passati in coincidenza attraverso due \emph{slab} di area uguale, pari a $S=L \cdot D = 1.83 \ \mbox{m} \cdot 0.2 \ \mbox{m} =0.366$ m$^2$, e distanti verticalmente di una quantità pari a $h$; tale parametro permette di scegliere la coppia di \emph{slab} di cui si vogliono valutare gli eventi in coincidenza, e nei conti che seguono si è assunto $h=0.08$ m, ovvero la distanza tra S1 e S2.

Il conto è stato effettuato considerando le \emph{slab} monodimensionali, bidimensionali e, per la simulazione, anche tridimensionali, cioè considerandone lo spessore; i risultati ottenuti sono in buon accordo fra di loro e con i dati sperimentali.
\subsection{Calcolo analitico del numero di coincidenze attese tra due \emph{slab} parallele}
\subsubsection{Caso unidimensionale}
La situazione è quella riportata nell'immagine che segue:
\newline
\begin{center}
\begin{tikzpicture}
	\newcommand\dz{0.2}
	\newcommand\dpmt{0.03}
	
	\node at (1,\dz/2) {S2};
	\draw[fill=black!25!white] (2,0-\dpmt) rectangle (10,\dz+\dpmt);
	\node at (6,-0.4) {0};
	\node at (8.7,-0.25) {x};

	\node at (1,1+\dz/2) {S1};
	\draw[fill=black!25!white] (2,1-\dpmt) rectangle (10,1+\dz+\dpmt);
	
	\draw (4,2) -- (8.7,\dz+\dpmt);
	\draw (8.7, \dz+\dpmt) -- (8.7,2);
	\draw (5, 1.63) -- (4.95, 1.73);
	\draw (5, 1.63) -- (4.9, 1.58);
	
	
	\draw [dashed] (2, 1-\dpmt) -- (8.7, \dz+\dpmt);
	\draw [dashed] (8.7, \dz+\dpmt) -- (10,1);
	
	\node at (8.4,0.65) {$\theta$};
	
	\draw (8.7, 0.5) arc [radius=0.4, start angle=90, end angle= 141];
	
	\draw (6,-\dpmt) -- (6,-0.2);
	\draw (8.7,-\dpmt) -- (8.7,-0.1);
	
	\draw [->] (1.5,-0.7) -- (11,-0.7);
\end{tikzpicture}
\end{center}

Il numero di eventi passati per la \emph{slab} superiore è $$N = \alpha \cdot T \cdot \int_{-\frac{L}{2}}^{\frac{L}{2}} dx \cdot \int_{-\frac{\pi}{2}}^{\frac{\pi}{2}} \cos ^2\theta d\theta $$e questo permette di fissare il valore della costante $$\alpha = \frac{2N}{\pi LT}$$
dove $T$ è il tempo di acquisizione. Ora, si osserva che su un punto sulla \emph{slab} inferiore corrispondente a $0<x<\frac{L}{2}$ arrivano gli eventi compresi fra due angoli $$\theta = -\arctan \bigg(\frac{\frac{L}{2}-x}{h}\bigg) \equiv \theta_1$$ e $$\theta = \arctan \bigg(\frac{\frac{L}{2}+x}{h}\bigg) \equiv \theta_2\;.$$ 
Data la simmetria del problema, si ottiene facilmente che la situazione è analoga per valori di $x$ negativi; si può allora scrivere che il numero di eventi passati per la \emph{slab} inferiore è $$N' =\alpha \cdot T \cdot \int_{-\frac{L}{2}}^{\frac{L}{2}} dx \int_{-\theta_1}^{\theta_2}  \cos ^2\theta d\theta$$ da cui si ricava subito $$\frac{N'}{N}=\frac{2}{\pi L} \int_{-\frac{L}{2}}^{\frac{L}{2}} dx \int_{-\theta_1}^{\theta_2}  \cos ^2 \theta d\theta \simeq 97.22 \%. $$ Tale valore è in ottimo accordo con la simulazione eseguita successivamente.

\subsubsection{Caso bidimensionale}
La situazione è ora quella riportata nelle due figure che seguono:
\newline

\begin{figure}[h]
\subfloat[Visione laterale.]{

\begin{tikzpicture}
	\newcommand\dz{0.2}
	\newcommand\dpmt{0.03}
	
	\node at (1.5,\dz/2) {S2};
	\draw[fill=black!25!white] (2,0-\dpmt) rectangle (8,\dz+\dpmt);
	
	\node at (1.5,1+\dz/2) {S1};
	\draw[fill=black!25!white] (2,1-\dpmt) rectangle (8,1+\dz+\dpmt);
	
	\draw (2,\dz+\dpmt) -- (2,1+\dz+\dpmt+0.7);
	\draw (2, \dz+\dpmt) -- (8.3,1+\dz+\dpmt+0.4);
	
	\draw (6.45,1+\dz+\dpmt) circle[radius=2pt];
	\fill (6.45,1+\dz+\dpmt) circle[radius=2pt];
	
	\draw (7.5,1+\dz+\dpmt+0.21) -- (7.55,1+\dz+\dpmt+0.29);
	\draw (7.5,\dz+\dpmt+1.21) -- (7.6,1+\dz+\dpmt+0.195);
	
	\node at (2.52,0.75) {$\theta_\text{max}$};
	
	\draw (2.25,0.1+\dz+\dpmt) arc [radius=0.25, start angle=0, end angle= 85];
		
	\draw [->] (1,-0.5) -- (1,1+\dz+\dpmt+1);
	
	\node at (1.35,1+\dz+\dpmt+0.7) {z};
\end{tikzpicture}}
\hspace{0.4cm}
\subfloat[Visione dall'alto.]{
\begin{tikzpicture}
	\newcommand\dz{0.2}
	\newcommand\dpmt{0.03}
	
	\draw[fill=black!25!white] (2,0.5) rectangle (6.6,1+\dz+\dpmt+0.5);
		
	\draw (2,0.8) -- (5,1.3);
	\draw (5,1.3) circle[radius=2pt];
	\fill (5,1.3) circle[radius=2pt];
	
	\draw (2,0.8) -- (7,0.8);
	
	\node at (4,0.95) {$\phi$};
	
	\draw (3.8,0.8) arc [radius=0.8, start angle=0, end angle= 21];
	
	\draw [->] (1,0) -- (7.75,0);
	\draw [->] (1,0) -- (1,2.7);
	
	\node at (7.25,0.3) {x};
	\node at (1.3,2.45) {y};	
	
\end{tikzpicture}}
\end{figure}

L'origine del sistema di riferimento è stata presa nel punto centrale della \emph{slab}; il pallino nero indica il punto di incidenza del raggio cosmico.

Come per il caso 1D, si fissa intanto la costante di proporzionalità, facendo attenzione al fatto che questa volta bisogna integrare sull'angolo solido, e $\theta\in(0,\frac{\pi}{2})$: 
$$N = \alpha \cdot T \cdot L \cdot D \cdot 2\pi \cdot \int_{0}^{\frac{\pi}{2}} \cos^2\theta  \sin\theta d\theta$$
che fornisce 
$$\alpha = \frac{3N}{2\pi TLD}.$$ 
A questo punto, per valutare il numero di eventi in coincidenza, si inizia l'analisi con $\phi\in(0, \frac{\pi}{2})$, che corrisponde all'evento mostrato nel disegno sopra. Un evento del genere passerà in coincidenza solo se la sua inclinazione è $$\theta \leq \arctan \bigg(\frac{\frac{L}{2}+x}{h\cdot \cos\phi}\bigg) \equiv \theta_1\;,$$ formula valida fino a quando il valore di $\phi$ non è tale da intersecare lo spigolo in basso a sinistra, ovvero $$\phi = \phi^{*} = \arctan \bigg(\frac{\frac{D}{2}+y}{\frac{L}{2}+x}\bigg)\;,$$ oltre il quale l'inclinazione massima è $$\theta = \arctan \bigg(\frac{\frac{D}{2}+y}{h\cdot \sin\phi}\bigg)  \equiv \theta_2\;.$$I raggi cosmici in coincidenza sono quindi dati da questo integrale, moltiplicato per 4, dato che $\phi\in(0, 2\pi)$ e il problema è simmetrico. Tramite gli strumenti di integrazione numerica del software Wolfram Mathematica si ottiene quindi:
\begin{multline*}
\frac{N'}{N}=4\cdot \alpha \cdot T\cdot \bigg( \int_{-\frac{L}{2}}^{\frac{L}{2}} dx \int_{-\frac{D}{2}}^{\frac{D}{2}} dy \int_0^{\phi^{*}} d\phi \int_0^{\theta_1} \cos^2\theta \sin\theta d\theta +\\
+ \int_{-\frac{L}{2}}^{\frac{L}{2}} dx \int_{-\frac{D}{2}}^{\frac{D}{2}} dy \int_{\phi^{*}}^{\frac{\pi}{2}} d\phi \int_0^{\theta_2} \cos^2\theta \sin\theta d\theta\bigg)\simeq 73.89 \%\;.
\end{multline*} 
Anche questo valore è in buon accordo con i risultati della simulazione 2D.

\subsection{Simulazione del numero di coincidenze}
L'approccio seguito per ogni simulazione è il medesimo: si sono innanzitutto generati $N$ eventi come set di numeri (di cardinalità dipendente dalla dimensione), corrispondenti ai raggi cosmici passanti per la \emph{slab} soprastante, e quindi, con semplici considerazioni geometriche, si sono contati gli eventi passati anche per la \emph{slab} sottostante. Il numero $N$ è stato stimato come il numero di muoni cosmici attesi sulla \emph{slab} soprastante in un tempo di $T=100$ s, ovvero $N=S\cdot T\cdot 130=4758$, ove $130$ è il numero di muoni per metro quadro per secondo atteso a livello del mare, secondo il \emph{Particle Data Group}; tuttavia, volendo stimare una percentuale, si è osservato che il numero assoluto $N$ di eventi è ininfluente, purché sufficiente ad avere una buona statistica (ovvero $N>100$).

Nel caso 3D si è invece variato leggermente l'approccio: dato che, di fatto, gli eventi incidenti sulla \emph{slab} superiore, considerata con il suo spessore di $2.7$ cm, possono attraversare il piano della \emph{slab} stessa in un qualunque punto, si sono generati $N=S\cdot T\cdot 130$ eventi su una superficie $S$ maggiore (e si sono effettuate simulazioni per diverse superfici), contando quelli passati per la \emph{slab} sopra, e la percentuale di questi passati in coincidenza anche attraverso la \emph{slab} sottostante.

\subsubsection{Caso unidimensionale}
L'evento è rappresentato da una coppia $(x, \theta)$, con $x$ distribuito uniformemente in (0, L) e $\theta$, ovvero l'angolo formato con la verticale, secondo il coseno quadrato in $(-\frac{\pi}{2}, \frac{\pi}{2})$; si simula quindi l'evento di raggio cosmico, andando a cercare la coincidenza verificando l'intersezione tra la retta passante per il punto $x$ e di pendenza nota.

Impostando $N = 4758$, si ottiene che la percentuale in coincidenza, per un valore casuale del seme, è del 97\%; variando il seme, tale valore non si discosta molto da quello indicato. Il grafico della simulazione è riportato insieme a quello delle altre simulazioni al termine del paragrafo.

\subsubsection{Caso bidimensionale}
In questo caso, l'evento è rappresentato da una quaterna numeri $(x,y,\theta,\phi)$, con $x$ distribuito uniformemente in (0, L), $y$ distribuito uniformemente in (0, D), $\theta$, l'angolo azimutale, distribuito secondo il coseno quadrato (moltiplicato per l'angolo solido) in $(0, \frac{\pi}{2})$, e $\phi$, l'angolo nel piano $xy$, distribuito uniformemente in $(0, 2\pi)$. Sfruttando sempre formule note di geometria analitica, si sono contati gli eventi in coincidenza, su un totale di 4758: in questo caso, si ottiene una percentuale intorno al 73.9\%.
Si noti che in questo caso la distribuzione dell'angolo azimuthale è del tipo $\cos ^2\theta \cdot \sin\theta$, dato che per avere il numero totale di eventi si deve integrare su tutto l'angolo solido.

\subsubsection{Caso tridimensionale}
Volendo stimare se il numero di eventi passanti lateralmente influisse in qualche modo sulla percentuale delle coincidenze, si è infine effettuata una simulazione tridimensionale. L'approccio, come già accennato, è leggermente diverso: l'evento è sempre rappresentato da una quaterna $(x,y,\theta,\phi)$, con però $x\in (-\mbox{square}, L+\mbox{square})$ e $y\in (-\mbox{square}, D+\mbox{square})$, dove \emph{square} è un parametro che permette di considerare la superficie allargata attorno alla \emph{slab}; generando il numero di eventi corrispondenti a queste diverse superfici, si considerano tutti gli eventi passanti per la \emph{slab} soprastante, con considerazioni geometriche, per quindi contare quanti, tra questi, siano passati anche per la \emph{slab} sottostante.
Si riempie, al variare del parametro \emph{square}, la seguente tabella:
\begin{table}[h]
	\caption{Risultati della simulazione 3D al variare della dimensione della superficie generatrice.}
	\centering
	\begin{tabular}{ccccc}
		\toprule
		Square [m]& Eventi totali&Eventi S1& \% coincidenza &\% laterali	\\	
		\midrule
		0.5	&	44148		&	5193	&	75.37\%	& 9\%	\\
		1	&	109538	&	5208	&	74.73\%	& 9\% \\
		2	&	318318	& 	5106	&	75.71\%	& 8\% \\
		5	&	1568658	&	5062	&	75.09\%	& 9\% \\
		10	&	5469958	&	4912	&	74.78\%	& 9\%\\
		
		\bottomrule
	\end{tabular}
	\label{3D_sim}
\end{table}

Si osserva che il numero di eventi passato lateralmente nella \emph{slab} superiore, dell'ordine del 9\%, non influisce particolarmente sulla percentuale degli eventi in coincidenza, che cresce leggermente fino al 75\%, ma non subisce notevoli variazioni rispetto al caso 2D; è anche possibile calcolare la percentuale di eventi in coincidenza passati lateralmente, ovvero chiedersi quanti degli eventi in coincidenza siano passati lateralmente nella \emph{slab} superiore: tale valore, non riportato in tabella per brevità, risulta sempre inferiore all'1\%. Inoltre, si nota che anche la dimensione della superficie su cui si generano gli eventi è ininfluente. Il grafico 3D riportato sotto considera un valore di \emph{square} $=10$ m.

Si noti che tale percentuale del 75\% è compatibile con quanto riportato in tabella \ref{coincidenze} dove si ha
\[\frac{\text{S1}\wedge\text{S2}}{\text{S1}}\sim70\%\ .\]

\begin{figure}[h]
	\centerline{
	\subfloat[1D]
	{\includegraphics[scale=0.5]{img/1D.pdf}} 
	\subfloat[2D]	
	{\includegraphics[scale=0.5]{img/2D.pdf}}}
	\centerline{\subfloat[3D]
		{\includegraphics[scale=0.8]{img/3D.pdf}}} 
	\caption{Grafici delle simulazioni.}
	\label{nDsim_plots}
\end{figure}


\section{Grafici}
\vspace{2cm}
\begin{figure}[h]
	\centerline{\includegraphics[scale=1.0]{img/time_tens_437.pdf}}
	\caption{Valori di tensione erogati in base al ritardo.}
	\label{time_tens}
\end{figure}
\begin{figure}[h]
	\centerline{\includegraphics[scale=1.0]{img/chan_time_437.pdf}}
	\caption{Canali corrispondenti ai ritardi impostati.}
	\label{chan_time}
\end{figure}
\begin{figure}[h]
	\centerline{\includegraphics[scale=0.9]{img/eff/eff1.pdf}}
\end{figure}
\begin{figure}[h]
	\centerline{\includegraphics[scale=0.9]{img/eff/eff2.pdf}}	
\end{figure}
\begin{figure}[h]
	\centerline{\includegraphics[scale=0.9]{img/eff/eff3.pdf}}	
\end{figure}
\begin{figure}[h]
	\centerline{\includegraphics[scale=0.9]{img/eff/eff4.pdf}}
\end{figure}
\begin{figure}[h]
	\centerline{\includegraphics[scale=0.9]{img/eff/eff5.pdf}}
\end{figure}
\begin{figure}[h]
	\centerline{\includegraphics[scale=0.9]{img/eff/eff6.pdf}}
\end{figure}
\begin{figure}[h]
	\centerline{\includegraphics[scale=0.9]{img/eff/eff7.pdf}}
\end{figure}
\begin{figure}[h]
	\centerline{\includegraphics[scale=0.9]{img/eff/eff8.pdf}}
\end{figure}
\begin{figure}[h]
	\centerline{\includegraphics[scale=0.9]{img/eff/eff9.pdf}}
\end{figure}
\begin{figure}[h]
	\centerline{\includegraphics[scale=0.9]{img/eff/eff10.pdf}}
\end{figure}
\begin{figure}[h]
	\centerline{\includegraphics[scale=0.9]{img/eff/eff11.pdf}}
\end{figure}
\begin{figure}[h]
	\centerline{\includegraphics[scale=0.9]{img/eff/eff12.pdf}}
\end{figure}
\begin{figure}[h]
	\centerline{\includegraphics[scale=0.9]{img/eff/eff13.pdf}}
\end{figure}
\begin{figure}[h]
	\centerline{\includegraphics[scale=0.9]{img/eff/eff14.pdf}}
\end{figure}
%\begin{figure}[h]
%	\centerline{
%		\subfloat
%		{\includegraphics[scale=0.5]{img/eff1.pdf}}
%		\subfloat
%		{\includegraphics[scale=0.5]{img/eff2.pdf}}}
%	\centerline{
%		\subfloat
%		{\includegraphics[scale=0.5]{img/eff3.pdf}}
%		\subfloat
%		{\includegraphics[scale=0.5]{img/eff4.pdf}}}
%\end{figure}
%\begin{figure}[h]
%	\centerline{
%		\subfloat
%		{\includegraphics[scale=0.5]{img/eff5.pdf}}
%		\subfloat
%		{\includegraphics[scale=0.5]{img/eff6.pdf}}}
%	\centerline{
%		\subfloat
%		{\includegraphics[scale=0.5]{img/eff7.pdf}}
%		\subfloat
%		{\includegraphics[scale=0.5]{img/eff8.pdf}}}
%\end{figure}
%\begin{figure}[h]
%	\centerline{
%		\subfloat
%		{\includegraphics[scale=0.5]{img/eff9.pdf}}
%		\subfloat
%		{\includegraphics[scale=0.5]{img/eff10.pdf}}}
%	\centerline{
%		\subfloat
%		{\includegraphics[scale=0.5]{img/eff11.pdf}}
%		\subfloat
%		{\includegraphics[scale=0.5]{img/eff12.pdf}}}
%\end{figure}
%\begin{figure}[h]
%	\centerline{
%		\subfloat
%		{\includegraphics[scale=0.5]{img/eff13.pdf}}
%		\subfloat
%		{\includegraphics[scale=0.5]{img/eff14.pdf}}}
%\end{figure}

%
%
%
%
\end{document}
