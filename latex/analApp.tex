\section{Previsioni teoriche}
Nei paragrafi successivi sono riportati due diversi approcci, uno analitico e uno computazionale, per tentare di stimare la percentuale di eventi passati in coincidenza attraverso due \emph{slab} di area uguale, pari a $S=L \cdot D = 1.83 \ \mbox{m} \cdot 0.2 \ \mbox{m} =0.366$ m$^2$, e distanti verticalmente di una quantità pari a $h$; tale parametro permette di scegliere la coppia di \emph{slab} di cui si vogliono valutare gli eventi in coincidenza, e nei conti che seguono si è assunto $h=0.08$ m, ovvero la distanza tra S1 e S2.

Il conto è stato effettuato considerando le \emph{slab} monodimensionali, bidimensionali e, per la simulazione, anche tridimensionali, cioè considerandone lo spessore; i risultati ottenuti sono in buon accordo fra di loro e con i dati sperimentali.
\subsection{Calcolo analitico del numero di coincidenze attese tra due \emph{slab} parallele}
\subsubsection{Caso unidimensionale}
La situazione è quella riportata nell'immagine che segue:
\newline
\begin{center}
\begin{tikzpicture}
	\newcommand\dz{0.2}
	\newcommand\dpmt{0.03}
	
	\node at (1,\dz/2) {S2};
	\draw[fill=black!25!white] (2,0-\dpmt) rectangle (10,\dz+\dpmt);
	\node at (6,-0.4) {0};
	\node at (8.7,-0.25) {x};

	\node at (1,1+\dz/2) {S1};
	\draw[fill=black!25!white] (2,1-\dpmt) rectangle (10,1+\dz+\dpmt);
	
	\draw (4,2) -- (8.7,\dz+\dpmt);
	\draw (8.7, \dz+\dpmt) -- (8.7,2);
	\draw (5, 1.63) -- (4.95, 1.73);
	\draw (5, 1.63) -- (4.9, 1.58);
	
	
	\draw [dashed] (2, 1-\dpmt) -- (8.7, \dz+\dpmt);
	\draw [dashed] (8.7, \dz+\dpmt) -- (10,1);
	
	\node at (8.4,0.65) {$\theta$};
	
	\draw (8.7, 0.5) arc [radius=0.4, start angle=90, end angle= 141];
	
	\draw (6,-\dpmt) -- (6,-0.2);
	\draw (8.7,-\dpmt) -- (8.7,-0.1);
	
	\draw [->] (1.5,-0.7) -- (11,-0.7);
\end{tikzpicture}
\end{center}

Il numero di eventi passati per la \emph{slab} superiore è $$N = \alpha \cdot T \cdot \int_{-\frac{L}{2}}^{\frac{L}{2}} dx \cdot \int_{-\frac{\pi}{2}}^{\frac{\pi}{2}} \cos ^2\theta d\theta $$e questo permette di fissare il valore della costante $$\alpha = \frac{2N}{\pi LT}$$
dove $T$ è il tempo di acquisizione. Ora, si osserva che su un punto sulla \emph{slab} inferiore corrispondente a $0<x<\frac{L}{2}$ arrivano gli eventi compresi fra due angoli $$\theta = -\arctan \bigg(\frac{\frac{L}{2}-x}{h}\bigg) \equiv \theta_1$$ e $$\theta = \arctan \bigg(\frac{\frac{L}{2}+x}{h}\bigg) \equiv \theta_2\;.$$ 
Data la simmetria del problema, si ottiene facilmente che la situazione è analoga per valori di $x$ negativi; si può allora scrivere che il numero di eventi passati per la \emph{slab} inferiore è $$N' =\alpha \cdot T \cdot \int_{-\frac{L}{2}}^{\frac{L}{2}} dx \int_{-\theta_1}^{\theta_2}  \cos ^2\theta d\theta$$ da cui si ricava subito $$\frac{N'}{N}=\frac{2}{\pi L} \int_{-\frac{L}{2}}^{\frac{L}{2}} dx \int_{-\theta_1}^{\theta_2}  \cos ^2 \theta d\theta \simeq 97.22 \%. $$ Tale valore è in ottimo accordo con la simulazione eseguita successivamente.

\subsubsection{Caso bidimensionale}
La situazione è ora quella riportata nelle due figure che seguono:
\newline

\begin{figure}[h]
\subfloat[Visione laterale.]{

\begin{tikzpicture}
	\newcommand\dz{0.2}
	\newcommand\dpmt{0.03}
	
	\node at (1.5,\dz/2) {S2};
	\draw[fill=black!25!white] (2,0-\dpmt) rectangle (8,\dz+\dpmt);
	
	\node at (1.5,1+\dz/2) {S1};
	\draw[fill=black!25!white] (2,1-\dpmt) rectangle (8,1+\dz+\dpmt);
	
	\draw (2,\dz+\dpmt) -- (2,1+\dz+\dpmt+0.7);
	\draw (2, \dz+\dpmt) -- (8.3,1+\dz+\dpmt+0.4);
	
	\draw (6.45,1+\dz+\dpmt) circle[radius=2pt];
	\fill (6.45,1+\dz+\dpmt) circle[radius=2pt];
	
	\draw (7.5,1+\dz+\dpmt+0.21) -- (7.55,1+\dz+\dpmt+0.29);
	\draw (7.5,\dz+\dpmt+1.21) -- (7.6,1+\dz+\dpmt+0.195);
	
	\node at (2.52,0.75) {$\theta_\text{max}$};
	
	\draw (2.25,0.1+\dz+\dpmt) arc [radius=0.25, start angle=0, end angle= 85];
		
	\draw [->] (1,-0.5) -- (1,1+\dz+\dpmt+1);
	
	\node at (1.35,1+\dz+\dpmt+0.7) {z};
\end{tikzpicture}}
\hspace{0.4cm}
\subfloat[Visione dall'alto.]{
\begin{tikzpicture}
	\newcommand\dz{0.2}
	\newcommand\dpmt{0.03}
	
	\draw[fill=black!25!white] (2,0.5) rectangle (6.6,1+\dz+\dpmt+0.5);
		
	\draw (2,0.8) -- (5,1.3);
	\draw (5,1.3) circle[radius=2pt];
	\fill (5,1.3) circle[radius=2pt];
	
	\draw (2,0.8) -- (7,0.8);
	
	\node at (4,0.95) {$\phi$};
	
	\draw (3.8,0.8) arc [radius=0.8, start angle=0, end angle= 21];
	
	\draw [->] (1,0) -- (7.75,0);
	\draw [->] (1,0) -- (1,2.7);
	
	\node at (7.25,0.3) {x};
	\node at (1.3,2.45) {y};	
	
\end{tikzpicture}}
\end{figure}

L'origine del sistema di riferimento è stata presa nel punto centrale della \emph{slab}; il pallino nero indica il punto di incidenza del raggio cosmico.

Come per il caso 1D, si fissa intanto la costante di proporzionalità, facendo attenzione al fatto che questa volta bisogna integrare sull'angolo solido, e $\theta\in(0,\frac{\pi}{2})$: 
$$N = \alpha \cdot T \cdot L \cdot D \cdot 2\pi \cdot \int_{0}^{\frac{\pi}{2}} \cos^2\theta  \sin\theta d\theta$$
che fornisce 
$$\alpha = \frac{3N}{2\pi TLD}.$$ 
A questo punto, per valutare il numero di eventi in coincidenza, si inizia l'analisi con $\phi\in(0, \frac{\pi}{2})$, che corrisponde all'evento mostrato nel disegno sopra. Un evento del genere passerà in coincidenza solo se la sua inclinazione è $$\theta \leq \arctan \bigg(\frac{\frac{L}{2}+x}{h\cdot \cos\phi}\bigg) \equiv \theta_1\;,$$ formula valida fino a quando il valore di $\phi$ non è tale da intersecare lo spigolo in basso a sinistra, ovvero $$\phi = \phi^{*} = \arctan \bigg(\frac{\frac{D}{2}+y}{\frac{L}{2}+x}\bigg)\;,$$ oltre il quale l'inclinazione massima è $$\theta = \arctan \bigg(\frac{\frac{D}{2}+y}{h\cdot \sin\phi}\bigg)  \equiv \theta_2\;.$$I raggi cosmici in coincidenza sono quindi dati da questo integrale, moltiplicato per 4, dato che $\phi\in(0, 2\pi)$ e il problema è simmetrico. Tramite gli strumenti di integrazione numerica del software Wolfram Mathematica si ottiene quindi:
\begin{multline*}
\frac{N'}{N}=4\cdot \alpha \cdot T\cdot \bigg( \int_{-\frac{L}{2}}^{\frac{L}{2}} dx \int_{-\frac{D}{2}}^{\frac{D}{2}} dy \int_0^{\phi^{*}} d\phi \int_0^{\theta_1} \cos^2\theta \sin\theta d\theta +\\
+ \int_{-\frac{L}{2}}^{\frac{L}{2}} dx \int_{-\frac{D}{2}}^{\frac{D}{2}} dy \int_{\phi^{*}}^{\frac{\pi}{2}} d\phi \int_0^{\theta_2} \cos^2\theta \sin\theta d\theta\bigg)\simeq 73.89 \%\;.
\end{multline*} 
Anche questo valore è in buon accordo con i risultati della simulazione 2D.

\subsection{Simulazione del numero di coincidenze}
L'approccio seguito per ogni simulazione è il medesimo: si sono innanzitutto generati $N$ eventi come set di numeri (di cardinalità dipendente dalla dimensione), corrispondenti ai raggi cosmici passanti per la \emph{slab} soprastante, e quindi, con semplici considerazioni geometriche, si sono contati gli eventi passati anche per la \emph{slab} sottostante. Il numero $N$ è stato stimato come il numero di muoni cosmici attesi sulla \emph{slab} soprastante in un tempo di $T=100$ s, ovvero $N=S\cdot T\cdot 130=4758$, ove $130$ è il numero di muoni per metro quadro per secondo atteso a livello del mare, secondo il \emph{Particle Data Group}; tuttavia, volendo stimare una percentuale, si è osservato che il numero assoluto $N$ di eventi è ininfluente, purché sufficiente ad avere una buona statistica (ovvero $N>100$).

Nel caso 3D si è invece variato leggermente l'approccio: dato che, di fatto, gli eventi incidenti sulla \emph{slab} superiore, considerata con il suo spessore di $2.7$ cm, possono attraversare il piano della \emph{slab} stessa in un qualunque punto, si sono generati $N=S\cdot T\cdot 130$ eventi su una superficie $S$ maggiore (e si sono effettuate simulazioni per diverse superfici), contando quelli passati per la \emph{slab} sopra, e la percentuale di questi passati in coincidenza anche attraverso la \emph{slab} sottostante.

\subsubsection{Caso unidimensionale}
L'evento è rappresentato da una coppia $(x, \theta)$, con $x$ distribuito uniformemente in (0, L) e $\theta$, ovvero l'angolo formato con la verticale, secondo il coseno quadrato in $(-\frac{\pi}{2}, \frac{\pi}{2})$; si simula quindi l'evento di raggio cosmico, andando a cercare la coincidenza verificando l'intersezione tra la retta passante per il punto $x$ e di pendenza nota.

Impostando $N = 4758$, si ottiene che la percentuale in coincidenza, per un valore casuale del seme, è del 97\%; variando il seme, tale valore non si discosta molto da quello indicato. Il grafico della simulazione è riportato insieme a quello delle altre simulazioni al termine del paragrafo.

\subsubsection{Caso bidimensionale}
In questo caso, l'evento è rappresentato da una quaterna numeri $(x,y,\theta,\phi)$, con $x$ distribuito uniformemente in (0, L), $y$ distribuito uniformemente in (0, D), $\theta$, l'angolo azimutale, distribuito secondo il coseno quadrato (moltiplicato per l'angolo solido) in $(0, \frac{\pi}{2})$, e $\phi$, l'angolo nel piano $xy$, distribuito uniformemente in $(0, 2\pi)$. Sfruttando sempre formule note di geometria analitica, si sono contati gli eventi in coincidenza, su un totale di 4758: in questo caso, si ottiene una percentuale intorno al 73.9\%.
Si noti che in questo caso la distribuzione dell'angolo azimuthale è del tipo $\cos ^2\theta \cdot \sin\theta$, dato che per avere il numero totale di eventi si deve integrare su tutto l'angolo solido.

\subsubsection{Caso tridimensionale}
Volendo stimare se il numero di eventi passanti lateralmente influisse in qualche modo sulla percentuale delle coincidenze, si è infine effettuata una simulazione tridimensionale. L'approccio, come già accennato, è leggermente diverso: l'evento è sempre rappresentato da una quaterna $(x,y,\theta,\phi)$, con però $x\in (-\mbox{square}, L+\mbox{square})$ e $y\in (-\mbox{square}, D+\mbox{square})$, dove \emph{square} è un parametro che permette di considerare la superficie allargata attorno alla \emph{slab}; generando il numero di eventi corrispondenti a queste diverse superfici, si considerano tutti gli eventi passanti per la \emph{slab} soprastante, con considerazioni geometriche, per quindi contare quanti, tra questi, siano passati anche per la \emph{slab} sottostante.
Si riempie, al variare del parametro \emph{square}, la seguente tabella:
\begin{table}[h]
	\caption{Risultati della simulazione 3D al variare della dimensione della superficie generatrice.}
	\centering
	\begin{tabular}{ccccc}
		\toprule
		Square [m]& Eventi totali&Eventi S1& \% coincidenza &\% laterali	\\	
		\midrule
		0.5	&	44148		&	5193	&	75.37\%	& 9\%	\\
		1	&	109538	&	5208	&	74.73\%	& 9\% \\
		2	&	318318	& 	5106	&	75.71\%	& 8\% \\
		5	&	1568658	&	5062	&	75.09\%	& 9\% \\
		10	&	5469958	&	4912	&	74.78\%	& 9\%\\
		
		\bottomrule
	\end{tabular}
	\label{3D_sim}
\end{table}

Si osserva che il numero di eventi passato lateralmente nella \emph{slab} superiore, dell'ordine del 9\%, non influisce particolarmente sulla percentuale degli eventi in coincidenza, che cresce leggermente fino al 75\%, ma non subisce notevoli variazioni rispetto al caso 2D; è anche possibile calcolare la percentuale di eventi in coincidenza passati lateralmente, ovvero chiedersi quanti degli eventi in coincidenza siano passati lateralmente nella \emph{slab} superiore: tale valore, non riportato in tabella per brevità, risulta sempre inferiore all'1\%. Inoltre, si nota che anche la dimensione della superficie su cui si generano gli eventi è ininfluente. Il grafico 3D riportato sotto considera un valore di \emph{square} $=10$ m.

Si noti che tale percentuale del 75\% è compatibile con quanto riportato in tabella \ref{coincidenze} dove si ha
\[\frac{\text{S1}\wedge\text{S2}}{\text{S1}}\sim70\%\ .\]

\begin{figure}[h]
	\centerline{
	\subfloat[1D]
	{\includegraphics[scale=0.5]{img/1D.pdf}} 
	\subfloat[2D]	
	{\includegraphics[scale=0.5]{img/2D.pdf}}}
	\centerline{\subfloat[3D]
		{\includegraphics[scale=0.8]{img/3D.pdf}}} 
	\caption{Grafici delle simulazioni.}
	\label{nDsim_plots}
\end{figure}

